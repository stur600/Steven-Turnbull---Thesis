\chapter*{Participant Information}

\section{Lily}
Lily is a female, second year undergraduate student who identifies as M\={a}ori and P\={a}keh\={a}. She was taking courses in biology, chemistry and physics, and she chose science, not only because she got good grades in it, but also because her family motivated her to take the subject. Lily sees herself as a ``culturally-involved'' person, and she believes this aligns well with her choice to pursue medicine at university as she hopes to help the well-being of her people. Her parents did not attend university, but her father got an education when he worked for the police and regretted dropping out of high school early. While her parents did not put pressure on her to ``go in any certain direction'' or enter into science explicitly, they did want to make sure she was aspiring for something and not just ``settling''. She attended a small, rural high school prior to university study, and took part in the Hikitia Te Ora course, a preparatory year for M\={a}ori and Pacific students going into first year health science. 

\section{Renee}
Renee is a female, second year undergraduate student who identifies as P\={a}keh\={a}. She attended a public, rural single sex girls school, and has always wanted to be in the police. She is studying physics at university just because she is interested in the subject and sees herself as a naturally curious person. She was originally enrolled in an arts degree but switched to a conjoint with physics after being inspired by a senior physics lecturer. While a high achiever, she felt that people saw her as a ``joke student'' and did not expect her to go on to university study. Renee is the first in her family to attend university, and does not have family who work as a scientist or in a job that uses science. She  and loves sharing knowledge with her friends and family: ``I can't talk to my parents and be like why does this work because they don't know. So I'm like if I can figure it out for myself and gain the knowledge myself then that's cool. Teach my kids one day and maybe make them more interested in the world.''

\section{Belvia}
Belvia is a female, fourth year undergraduate student who identifies as Indian. Belvia's family emmigrated to New Zealand from India when she was 9 years old. Belvia says that her parents, who both went to university, were motivated to come to New Zealand to provide their children with access to university study. Belvia felt a lot of pressure from her parents to perform well academically and study at university. originally intended to study engineering, but did not meet the entry requirements. She changed her major to commerce, but after two weeks decided that was not for her. Belvia then went in to computer science following the recommendation of a ``friend of a friend'', but went into the degree not knowing much about what was involved: ``I didn’t know anyone who did this degree and I was so confused... I was so confused in first year I was oh what is happening.'' Prior to university, Belvia attended a small, single sex private school, in part because her father was worried about ``drugs and stuff'' in public school. Belvia enjoyed her time at her high school, which she felt was ``like a second family''. She recalled being an extrovert at high school when she was surrounded by her friends, but feels ``isolated'' at university where students keep to themselves. She is unsure of what ``kind of person'' (``I don’t view myself as an art person, I don't view myself as a maths person'') she is, but feels that this will come once she has graduated and got a job that she likes.


\section{Chloe}
Chloe is a female, second year undergaduate student who identifies as M\={a}ori. She comes from a highly academic family, with three generations of her family attending university. She spent an extended period in the hospital while she was young, and decided that she wanted to work as a doctor, which her parents ``supported that times a hundred.'' While she also wanted to be a dancer, that goal was ``crushed'' with the expectation that she would go to university. She has a passion for mental health and aspires to be a clinical psychologist. She originally pursued biomed and health at a different university, but changed her path after experiencing an unhealthy competitive culture. She is part of a ``very strong minded successful M\={a}ori family'', who have expectations for Chloe to continue to break the cycle of oppression. She is now studying psychology, which she feels is ``not sufficing'' with regards to her own and her family's expectations,as it is a ``generic'' degree. Chloe feels she is expected to go overseas to complete post-graduate qualifications, maybe at an ivy league institution. Chloe feels a ``big responsibility'' to live up to her family's expectation and continue ``to make a positive influence in M\={a}ori society''.




\section{Lucy}
Lucy is a female first year undergraduate student who identifies as P\={a}keh\={a}. Lucy spent most of her life in Switzerland, but her parents lived in New Zealand for an extended time before Lucy was born. Both of Lucy's parents are professionals who work in science. She has wanted to study physics since she was young: ``I told my parents when I was seven that I wanted to be a physicist and I always stuck to it''. Her current ``dream goal'' is to become a sailor, and this influenced her to continue to study physics with exercise science. Lucy attended a prestigious private school prior to university where she gained an IB qualification. 

\section{Sean}
Sean is a male first year undergraduate student who identifies as M\={a}ori and Nieuen. He attended a public, single-sex boys school in Auckland prior to university. Sean is studying courses in biology and chemistry, with the goal of becoming a surgeon. His aspirations are tied to his religious views, where he is encouraged to be ``more Christ-like'', and his cultural identity. He sees things like heart disease as a major concern for M\={a}ori and Pacific populations: ``maybe being able to help people out with heart disease would be a good thing, but support to help my family out''. Prior to university, he undertook a mission with his church which helped him develop self-discipline. Sean's mother has a science degree, while his sister just graduated with a science degree in biomed and is undertaking postgraduate study. Sean's parents always expected him to do well in school, but made sure he had experiences of doing non-academic jobs, such as working in a factory. 

\section{Mark}
Mark is a first year undergraduate student who identifies as P\={a}keh\={a} and M\={a}ori. He attended a private school where he underwent CIE qualifications and achieved a maximum rank score. At his high school, Mark felt that ``it was sort of like an assumption that everyone was going to uni ''. Mark is now studying engineering at university, although he is unsure of what being an engineer actually looks like. He commented that he could have enjoyed being an English teacher, but he felt pressure to do engineering because he is ``literally able to do it and in a more mental academic sense I will get through''. Marks's father, who did a pharmaceutical degree at university, advised Mark to not go into pharmacy because he felt it is boring, but to do something that he enjoyed. Mark did recall pressure from his Mum, who holds expectations for Mark being the ``academic child''. Mark felt that he does not really fit the typical stereotype of a scientist both appearance wise (``I don’t fit the whole archetype of like nerdy engineering kid'') and when he compares himself to some of his friends who are ``really studious''. With that being said, Mark ran the after school science club for his school, and is passionate about science. 


\section{Jay}
Jay is a male, first year undergraduate who identifies as European. He is originally from South Africa, and moved to New Zealand when he was 9 years old. He is now studying computer science with the goal of working in a mix of finance and computers. Both of Jay's parents went to university, with Jay's father completing two degrees in electrical engineering and computer science. While Jay's parents ensured that Jay progressed academically by teaching him outside of school hours, they had the ``philosophy that whatever you want to do to make you happy would be the best to do''. Prior to university, Jay attended a high school with no designated curriculum, but switched to a public school after him and his parents decided that it was not a good fit. 

\section{Alicia}
Alicia is a female, first year undergraduate student who identifies as Chinese. She went to high school and completed a degree in commerce in China before moving to New Zealand to study computer science as a mature student. She did not like commerce, and stated that it was a ``family choice''. As Alicia is the first in her family to attend university, she says that this choice was informed mainly by stereotypes held by her family (``girls need to do girls job''). Even though her parents did not attend university, Alicia found that the her parents gave her an environment that facilitated her interest in science. While she has a good relationship with her mother, she left China in part to ``run away'' from family pressure and a working culture that competitive and ``not that friendly to women''. Alicia decided to study computer science at university in New Zealand after working with an IT company and being ``immersed'' in that environment. She feels that computer science aligns with her ability to problem solve, and aspires to be a data scientist working in silicon valley.


\section{Aubrey}
Aubrey is a female, third year undergraduate student who identifies as P\={a}keh\={a} and M\={a}ori. She is studying biology and is the first in her family to attend university. While she sees herself as an ``artsy'' person, she has always been interested in science. She attributes the start of this interest to science books she read when she was around 9 years old. She went to a school ``a poor area'', and did not think she ``would get into university''. She enjoyed science at high school and recalls having teachers who were enthusiastic and excited about teaching, and felt they expected her to go on to university. After high school, Aubrey ended up in Auckland after her mother found her a administration job. Aubrey found that this job ``is really not for me'' and wanted something that was more stimulating. She chose to study biology at university as she did not do as well in chemistry, and did not know what jobs were available in physics. She would like to work in forestry science: ``I just love trees and I grew up around forestry''. Aubrey feels she is quite shy, and would struggle to talk to her lecturers: ``I’m just nervous I don’t know how to approach anyone about that''. Her parents are both employed in the social work space, but Aubrey feels like that would not be ``very fulfilling'' for her. She is more motivated to help solve environmental issues. While she was one of only a few students from her school who went to study science at university, Aubrey is able to talk to her friends and her partner about what she studies. She is thankful for the support she received from her family, friends and teachers. 

 

\section{Stephen}
Stephen is a trans male, first year undergraduate student who identifies as P\={a}keh\={a} and M\={a}ori. He is studying physics, and he has always wanted to work for NASA (aspiring to be a rocket physicist or chemist). This may have been informed by Stephen's mother (who went to university, but not in science) talking about the moon-landings and space while he was growing up. He feels a high level of affinity with science, and feels that his friends see him as a ``nerd'': `` everyone is like oh you are such a nerd and I’m like yeah that's me''. Prior to university, Stephen attended a private school and then moved to a public rural school for his last two years of high school. He found that private school ``spoon-feeds'' students to go to university, while most of his peers at rural school went into vocational careers. Stephen has found the transition to university study difficult as he misses rural life, and few of his friends from high school moved to Auckland. Stephen has also had to navigate discussing his gender identity with his new peers, and this can be anxiety provoking: ``it is like quite confronting and nerve wracking for me to open up to people about me''. With that being said, Stephen is proud of himself for making new friends, and he feels ``blessed'' to have friends and teachers who support him and his identity.  

\section{Hakeem}
Hakeem is a male, fourth year undergraduate student who identifies as P\={a}keh\={a} and Tongan. He is now studying computer science, after switching from his original engineering degree. This swap was motivated partly because he felt ``naturally better'' at the software side of engineering than the hardware side, and partly because he was worried he would not get enough work hours to complete his engineering degree (engineering students are required to complete 800 hours work experience). Hakeem found that many engineering students are able to attain hours due to family connections in the field, which he does not have. Hakeem's parents did not attend university, and they did not get involved in his academic life. Hakeem, who also played rugby to a high standard, felt like his family were more interested in his sporting achievements than his academic ones: ``I felt like during high school especially I wasn't guided at all by anyone really. My parents were non-existent in my high school life they didn't really put any attention towards where I was going''. He went into the first year of engineering as ``one of the only Islanders in the class'' and sometimes felt anxious and under pressure to perform. After experiencing adversity in his family life, he moved into his own place with his girlfriend and is feeling ``happier and more content''. He attributes much of his current motivation and success to wanted to provide for his girlfriend, while he also acknowledges the importance of having ``consistent'' friends, and ``very good'' high school teachers. Hakeem wants to work as a data scientist, and is intrinsically motivated to succeed: ``I'm self-aware of how much I enjoy it and that is why I kind of do it''.  

\section{Patrick}
Patrick is a male, second year undergraduate student who identifies as M\={a}ori. He attended a single sex boys school, but dropped out of high school when he was 16 years old. He received mixed support at high school with one teacher telling him he should drop out, but with the dean trying to encourage him to stay. He originally intended to join the army, following in the steps of much of his family, but ended up working in a saw mill with his father. He draws upon this experience as a prime motivator for why he went to university: ``it was a really hard job and I hated it. So that is probably why I ended up coming to uni.'' He is now studying computer science as he likes technology and feels there is good opportunity to earn money. Patrick's parents did not go to university, but he talks with his Dad about science and his degree. Patrick sees stereotypes facing M\={a}ori as a motivating factor: ``that is kind of the reason why I want to study science as well because a lot of people think oh you are Maori, you are Polynesian, you are not smart enough to do stuff like that.'' He has found teachers at university ``way more helpful'' than those at his high school, and spends much of his university days at Tu\={a}kana, a university support service for M\={a}ori and Pasifika students.  

\section{Longolongo}
Longolongo  is a male, third year undergraduate who identifies as Tongan. He attended high school in Tonga, and worked as a nurse prior to attending university, where he is studying biology. Longolongo wants to ``make a mark back in Tonga'' and has aspirations to set up a forensics department. Longolongo described himself as a ``a bit like weird twink'', who many people in Tonga find rebellious. He encountered opposition to his lifestyle while growing up in Tonga and working as a nurse but he stayed true to himself. Longolongo's parents did not attend high school or university, but they gave him support to be himself in his everyday life: ``I was the only fem boy back in our village and my parents usually like don’t give a f like how I behave and stuff like that... They didn't want me to kind of like not follow myself I think, and back in Tonga people usually listen to others but my parents they didn't.'' Longolongo found the transition to and English speaking university difficult, and feels that given the high tuition he has to pay, there should be more support. 


\section{Diana}
Diana is female, first year undergraduate student who identifies as P\={a}keh\={a}. She is in her first year of studying chemistry at Auckland, while she is also enrolled in a diploma of languages. Diana spent the previous year studying geology in Wellington, but decided to change her degree as geology was not ``for her'' and she did not find it interesting. She originally chose geology as she thought it had good job prospects, paid well, and was easy (``okay geology how hard can it be''). Diana is the first in her family to attend university, her father works as a ``tradie'' and her mother works in IT. Diana recalled feeling ``pushed'' by her mother in particular to go into IT. She went to a single-sex girls school, where she felt that her teachers she felt like her teachers did not recognise her as being good at science, but saw her as ``kind of like the dumb one of the smart girls, or like the smart one of the other students''. 





