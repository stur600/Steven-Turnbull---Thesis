\chapter{Understanding Capital Accumulation in University Science}


\section{Introduction}
The goal of this chapter is to build a theoretical model, based on Bourdieu's sociology, that may help us understand students' experiences of university science education. More specifically, the model seeks to describe the reciprocal relationship shared between students' internal dispositions (habitus) and their social relationships (social capital). The following sections will describe, step by step, the manner in which social capital relates to habitus, and how habitus directs future accumulation of capital. Using examples from the field of university science education, this paper also describe how each of the six component parts of the theoretical model may be targeted to address equity issues in university science education.

\begin{figure}[ht]
\centering
\includegraphics[width=\textwidth]{HabitusSocCap_TheoreticalModel.png}
\caption{\label{fig:TheoreticalModel_C5}\textbf{Complete Theoretical Model}. The complete theoretical model. The model, informed by the described experiences of university science students, expands on the interaction between habitus and capital originally illustrated by \cite{Bourdieu1984}. The model provides a step by step description of how social capital translates into habitus transformation, and how habitus generates future capital accumulation. Step 1 refers to the initial availability of social capital for an individual. Step 2 refers to the value gained through leveraging social capital into forms of economic and cultural capital. Step 3 refers to how social capital is internalised by the individual. Step 4 refers to how habitus informs the acquisition of future social capital. 5 refers to factors outside of the individual that structure the field (by availability of capital and exchange value).}
\end{figure}

\section{Social Capital in the Field of University Science (Step 1)}
This section discusses the availability of social relationships that participants had in the field of university science (Figure \ref{fig:TheoreticalModel1_C5}). These relationships are categorised into three key groups, social capital through family connections, social capital through lecturers, and social capital through peers. 

\begin{figure}[h!]
\centering
\includegraphics[width=\textwidth]{HabitusSocCap_TheoreticalModel1.png}
\caption{\label{fig:TheoreticalModel1_C5}\textbf{Step 1}. The first step of the theoretical model. Step 1 refers to the initial availability of social capital for an individual.}
\end{figure}



\section{Leveraging Social Capital (Step 2)}
The value of social capital is derived from the economic and cultural capital that can be mobilised through connections with others (Figure \ref{fig:TheoreticalModel2_C5}). These resources may provide students with advantages in supporting their learning and providing student access to information about the field. 

\begin{figure}[ht]
\centering
\includegraphics[width=\textwidth]{HabitusSocCap_TheoreticalModel2.png}
\caption{\label{fig:TheoreticalModel2_C5}\textbf{Step 2}. The second step of the theoretical model. This step refers to the utility of social capital in mobilising cultural capital and economic capital.}
\end{figure}


\section{Internalising Capital (Step 3)}
The capital that students accrue in university science not only provides access to resources, but it can also influences the way that students see themselves in the field. Capital, which determines students position in the field, is internalised by students via their habitus  (Figure \ref{fig:TheoreticalModel3_C5}), which establishes the students disposition towards the field \cite{Bourdieu1992}. Students who hold high levels of social capital may feel an affinity with the field and see progression to university as a path already drawn out.

\begin{figure}[ht]
\centering
\includegraphics[width=\textwidth]{HabitusSocCap_TheoreticalModel3.png}
\caption{\label{fig:TheoreticalModel3_C5}\textbf{Step 3}. The third step of the theoretical model. This step refers to the manner in which capital is internalised by individuals via habitus.}
\end{figure}



\section{Doxa (Step 4)}
Outside of the of the social relationships that students hold, general discourses in society influence the development of habitus (Figure \ref{fig:TheoreticalModel4_C5}). 

\begin{figure}[ht]
\centering
\includegraphics[width=\textwidth]{HabitusSocCap_TheoreticalModel4.png}
\caption{\label{fig:TheoreticalModel4_C5}\textbf{Step 4}. The fourth step of the theoretical model. This step refers to the impact doxa (pervasive societal discourses) has on students' habitus.}
\end{figure}

\section{Generating Social Capital (Step 5)}
The future acquisition of social capital is informed by habitus (Figure \ref{fig:TheoreticalModel5_C5}). 
\begin{figure}[ht]
\centering
\includegraphics[width=\textwidth]{HabitusSocCap_TheoreticalModel5.png}
\caption{\label{fig:TheoreticalModel5_C5}\textbf{Step 5}. The fifth step of the theoretical model. This step refers to the manner in which students' habitus directs the future acquisition of social capital}
\end{figure}



\section{Institutional Habitus (Step 6)}
The cycle is impacted on by interventions operating in the field, which themselves are directed by an institutional habitus (Figure \ref{fig:TheoreticalModel6_C5}). 

\begin{figure}[ht]
\centering
\includegraphics[width=\textwidth]{HabitusSocCap_TheoreticalModel.png}
\caption{\label{fig:TheoreticalModel6_C5}\textbf{Step 6}. The sixth step of the theoretical model. This step refers to the manner in which institutions impact on the cycle}
\end{figure}



\section{Conclusion}


