\documentclass[11pt]{aucklandthesis}
%
% This is a template for University of Auckland theses.
%
% Written by Alistair Kwan, June 2016
% 
%
% Options:
%	10pt, 11pt, 12pt: size of main text
% 	examcopy: asserts confidentiality for examination copies
%	partial: thesis partial fulfils degree requirements
%	singlespace, onehalfspace, doublespace: line spacing
%	oneside: format for single-sided printing
%	draft: adds 'draft' and date to footer
%

%
% Add, delete or un-comment packages below as required.
%

\usepackage[utf8]{inputenc}
\usepackage[T1]{fontenc}

%\usepackage{graphicx} % for inserting graphics files
%\usepackage{appendix} % for appendices

%\usepackage{hyperref} % for formatting web addresses and other URLs
%\urlstyle{same} % try also tt, sf if this option doesn't produce clear enough output

% Readability options
%
%\usepackage{booktabs} % for table rules
%\usepackage{microtype} % for improved justification

% Typeface options — choose one if desired
% or choose a different typeface to accommmodate character sets
% as needed for East Asian and other languages.
%
% Consider compiling using the XeLaTeX engine if you have more extreme
% typeface needs, e.g. for multiple languages, or a need for symbols particular
% to a typeface.
%
% See also the LaTeX Symbols List at
% https://http://www.ctan.org/pkg/comprehensive
%
%\usepackage{mathptmx} % Times New Roman, including mathematics
%\usepackage{mathpazo} % Palatino with mathematics support
%\usepackage{fourier} % Utopia, a serif typeface with Fourier mathematics
%\usepackage{gentium} % a contemporary serif typeface
%\usepackage{libertine} % a softer-feeling serif typeface; also installs sans-serif font Biolinum
%\usepackage{fouriernc} % Century Schoolbook with Fourier maths
%\usepackage{mathpple} % Palatino with Fourier maths


% To set the sans serif font (for \sffamily):
%\usepackage[scaled]{helvet} % Nimbus, like Helvetica
%\usepackage{universalis} % Universalis
%\usepackage{avant} % URW Gothic, like Avant Garde
%\usepackage{PTSansNarrow}
%\usepackage{AlegreyaSans} % Alegreya Sans

% To set the mathematics font:
%\usepackage{eulervm} % Euler, based on a Zapf design

% To set the (usually monospaced) typewriter font:
%\usepackage[ttdefault=true]{AnonymousPro}
%\usepackage[scaled]{beramono}
%\usepackage{inconsolata}
%\usepackage{sourcecodepro}

%\usepackage{cjk} % for Chinese, Japanese, Korean

%\usepackage{tabularx} % For easier table formatting.

%\usepackage[nottoc]{tocbibind} % Controls the table of contents
%   nottoc: don't list table of contents inside itself
%   section: go as far as section-level headings

% Automated bibliography
%
%\usepackage[
%	style=authortitle, 
%	citestyle=authortitle,
%	backend=biber
%	]
%	{biblatex}
%bibliography{bibliography1.bib, bibliography2.bib} % Specify bibliography files 

\begin{document}

% ====================================================
%
% FRONTMATTER
%
% Arabic pagination, starting with the title page
% which is counted but not numbered
%
% ====================================================

% Specify the title page content
\title{Understanding Participation in Science Education in New Zealand}
\subtitle{A Mixed Methods, Bourdieusian Approach}
\author{Steven Martin Turnbull}
\degreesought{Doctor of Philosophy} 
\degreediscipline{Education}
\degreecompletionyear{2020}

% Print the title page
\maketitle

% Abstract, up to 350 words
%\chapter*{Abstract}
Words gonna go here
 % it's in a separate file

% Dedication (optional)
%\thesisdedication{Dedicated to grandma, and to grammar.}

% Preface and/or acknowledgements (optional)
%\chapter*{Acknowledgements}
 % it's in a separate file

% Contents, lists of tables and figures
\settocdepth{section} % choose chapter, section, subsection \cleardoublepage\tableofcontents
%\cleardoublepage\listoffigures
%\cleardoublepage\listoftables

% Glossary (optional)
%\chapter*{Glossary}.

% ====================================================
%
% MAINMATTER
%
% Include external chapter files here using
% the \input{} command
%
% If you run out of memory during compilation,
% switch some or all chapters to \include{} instead of \input{}, 
% but watch out for pagination problems.
%
% ====================================================

%\input{chapter1} % I hope that you have better titles than this
%\input{chapter2} 
%\input{chapter3} 

% ====================================================
%
% ENDMATTER
%
% Appendices and bibliography 
% Pagination arabic, re-starts at 1
%
% ====================================================
\cleardoublepage % start afresh on a new page
\setcounter{page}{1} % re-sets the page counter
%\appendixpage* % makes a page to mark beginning of appendices
% \input{appendix1} 

%\printbibliography[title={Works cited}, heading=bibintoc]

\end{document}